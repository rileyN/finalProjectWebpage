

\section*{\mbox{\hyperlink{namespacepybind11}{pybind11}} — Seamless operability between C++11 and Python}

\href{http://pybind11.readthedocs.org/en/master/?badge=master}{\texttt{ }} \href{http://pybind11.readthedocs.org/en/stable/?badge=stable}{\texttt{ }} \href{https://gitter.im/pybind/Lobby}{\texttt{ }} \href{https://travis-ci.org/pybind/pybind11}{\texttt{ }} \href{https://ci.appveyor.com/project/wjakob/pybind11}{\texttt{ }}

{\bfseries{\mbox{\hyperlink{namespacepybind11}{pybind11}}}} is a lightweight header-\/only library that exposes C++ types in Python and vice versa, mainly to create Python bindings of existing C++ code. Its goals and syntax are similar to the excellent \href{http://www.boost.org/doc/libs/1_58_0/libs/python/doc/}{\texttt{ Boost.\+Python}} library by David Abrahams\+: to minimize boilerplate code in traditional extension modules by inferring type information using compile-\/time introspection.

The main issue with Boost.\+Python—and the reason for creating such a similar project—is Boost. Boost is an enormously large and complex suite of utility libraries that works with almost every C++ compiler in existence. This compatibility has its cost\+: arcane template tricks and workarounds are necessary to support the oldest and buggiest of compiler specimens. Now that C++11-\/compatible compilers are widely available, this heavy machinery has become an excessively large and unnecessary dependency.

Think of this library as a tiny self-\/contained version of Boost.\+Python with everything stripped away that isn\textquotesingle{}t relevant for binding generation. Without comments, the core header files only require $\sim$4K lines of code and depend on Python (2.\+7 or 3.\+x, or Py\+Py2.\+7 $>$= 5.\+7) and the C++ standard library. This compact implementation was possible thanks to some of the new C++11 language features (specifically\+: tuples, lambda functions and variadic templates). Since its creation, this library has grown beyond Boost.\+Python in many ways, leading to dramatically simpler binding code in many common situations.

Tutorial and reference documentation is provided at \href{http://pybind11.readthedocs.org/en/master}{\texttt{ http\+://pybind11.\+readthedocs.\+org/en/master}}. \mbox{\hyperlink{struct_a}{A}} P\+DF version of the manual is available \href{https://media.readthedocs.org/pdf/pybind11/master/pybind11.pdf}{\texttt{ here}}.

\subsection*{Core features}

\mbox{\hyperlink{namespacepybind11}{pybind11}} can map the following core C++ features to Python


\begin{DoxyItemize}
\item Functions accepting and returning custom data structures per value, reference, or pointer
\item Instance methods and static methods
\item Overloaded functions
\item Instance attributes and static attributes
\item Arbitrary exception types
\item Enumerations
\item Callbacks
\item Iterators and ranges
\item Custom operators
\item Single and multiple inheritance
\item S\+TL data structures
\item Smart pointers with reference counting like {\ttfamily std\+::shared\+\_\+ptr}
\item Internal references with correct reference counting
\item C++ classes with virtual (and pure virtual) methods can be extended in Python
\end{DoxyItemize}

\subsection*{Goodies}

In addition to the core functionality, \mbox{\hyperlink{namespacepybind11}{pybind11}} provides some extra goodies\+:


\begin{DoxyItemize}
\item Python 2.\+7, 3.\+x, and Py\+Py (Py\+Py2.\+7 $>$= 5.\+7) are supported with an implementation-\/agnostic interface.
\item It is possible to bind C++11 lambda functions with captured variables. The lambda capture data is stored inside the resulting Python function object.
\item \mbox{\hyperlink{namespacepybind11}{pybind11}} uses C++11 move constructors and move assignment operators whenever possible to efficiently transfer custom data types.
\item It\textquotesingle{}s easy to expose the internal storage of custom data types through Pythons\textquotesingle{} buffer protocols. This is handy e.\+g. for fast conversion between C++ matrix classes like Eigen and Num\+Py without expensive copy operations.
\item \mbox{\hyperlink{namespacepybind11}{pybind11}} can automatically vectorize functions so that they are transparently applied to all entries of one or more Num\+Py array arguments.
\item Python\textquotesingle{}s slice-\/based access and assignment operations can be supported with just a few lines of code.
\item Everything is contained in just a few header files; there is no need to link against any additional libraries.
\item Binaries are generally smaller by a factor of at least 2 compared to equivalent bindings generated by Boost.\+Python. \mbox{\hyperlink{struct_a}{A}} recent \mbox{\hyperlink{namespacepybind11}{pybind11}} conversion of Py\+Rosetta, an enormous Boost.\+Python binding project, \href{http://graylab.jhu.edu/RosettaCon2016/PyRosetta-4.pdf}{\texttt{ reported}} a binary size reduction of {\bfseries{5.\+4x}} and compile time reduction by {\bfseries{5.\+8x}}.
\item Function signatures are precomputed at compile time (using {\ttfamily constexpr}), leading to smaller binaries.
\item With little extra effort, C++ types can be pickled and unpickled similar to regular Python objects.
\end{DoxyItemize}

\subsection*{Supported compilers}


\begin{DoxyEnumerate}
\item Clang/\+L\+L\+VM 3.\+3 or newer (for Apple Xcode\textquotesingle{}s clang, this is 5.\+0.\+0 or newer)
\item G\+CC 4.\+8 or newer
\item Microsoft Visual Studio 2015 Update 3 or newer
\item Intel C++ compiler 17 or newer (16 with \mbox{\hyperlink{namespacepybind11}{pybind11}} v2.\+0 and 15 with \mbox{\hyperlink{namespacepybind11}{pybind11}} v2.\+0 and a \href{https://github.com/pybind/pybind11/issues/276}{\texttt{ workaround}})
\item Cygwin/\+G\+CC (tested on 2.\+5.\+1)
\end{DoxyEnumerate}

\subsection*{About}

This project was created by \href{http://rgl.epfl.ch/people/wjakob}{\texttt{ Wenzel Jakob}}. Significant features and/or improvements to the code were contributed by Jonas Adler, Sylvain Corlay, Trent Houliston, Axel Huebl, @hulucc, Sergey Lyskov Johan Mabille, Tomasz Miąsko, Dean Moldovan, Ben Pritchard, Jason Rhinelander, Boris Schäling, Pim Schellart, Ivan Smirnov, and Patrick Stewart.

\subsubsection*{License}

\mbox{\hyperlink{namespacepybind11}{pybind11}} is provided under a B\+S\+D-\/style license that can be found in the {\ttfamily L\+I\+C\+E\+N\+SE} file. By using, distributing, or contributing to this project, you agree to the terms and conditions of this license. 